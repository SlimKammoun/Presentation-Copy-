 	\documentclass[english,xcolor=table]{beamer}
\usepackage[round]{natbib}

\usepackage{pgfpages}
\mode<presentation> {
  \usetheme{Madrid}

  \setbeamercovered{transparent}
}
\newtheorem{proposition}[theorem]{Proposition}
\newtheorem{corollaire}[theorem]{Corollaire}
\usepackage{calc}

\usepackage[utf8]{inputenc}
\usepackage[T1]{fontenc}
\usepackage{babel}
\usepackage{times}
\usepackage[T1]{fontenc}
\usepackage{tikz}
\usepackage{amsfonts}
\usepackage{pgfplots}
\pgfplotsset{compat=newest}
\usepackage[vcentermath]{youngtab}
%\pgfdeclareimage[height=0.5cm]{le-logo}{logo-irisa}
%\logo{\pgfuseimage{le-logo}}
\setbeamertemplate{footline}[frame number]
\newtheorem{conjecture}[theorem]{Conjecture}

\usepackage{fourier, heuristica}
\usepackage{array, booktabs}
\usepackage{graphicx}
\usepackage{xcolor}
\usepackage{tikz}
\usetikzlibrary{automata, positioning}
\usepackage{caption}
\DeclareCaptionFont{blue}{\color{LightSteelBlue3}}

\newcommand{\foo}{\color{LightSteelBlue3}\makebox[0pt]{\textbullet}\hskip-0.5pt\vrule width 1pt\hspace{\labelsep}}


%%%%%%%%%%%%%%%%%%%%%%%%%%%
\title{Universality for random permutations}

\subtitle {Séminaire MEGA}
\author 
{ \large{Mohamed Slim Kammoun}
\\  \ \\ \large{Supervisors:  Mylène Maïda  and Adrien Hardy}
\\ \ \\ {Laboratoire Paul Painlevé, Université de Lille}
}
\date {  January 11, 2019}
\titlegraphic{
\includegraphics[height=0.9cm]{l1}
   \includegraphics[height=0.9cm]{0}
   \includegraphics[height=0.9cm]{l6}
   
}

\usepackage{xcolor}
\newcommand\ytl[2]{
\parbox[b]{5em}{\hfill{\color{cyan}\bfseries\sffamily #1}~$\cdots\cdots$~}\makebox[0pt][c]{$\bullet$}\vrule\quad \parbox[c]{5.5cm}{\vspace{7pt}\color{red!40!black!80}\raggedright\sffamily #2.\\[7pt]}\\[-3pt]}
\newcommand{\s}{\mathfrak{S}_n}

\begin{document}

\begin{frame}
  \titlepage  
\end{frame}



\section*{Introduction}
\begin{frame}{Universality}
 \begin{overprint}
\onslide<1>
GUE :
\begin{table}[]
\begin{tabular}{|l|l|}
\hline
                   & GUE              \\ \hline
Biggest particule  & T.W              \\ \hline
Edge               & Soft edge (Airy) \\ \hline
Global convergence & Semi circular    \\ \hline
Fluctuations       & Gaussian         \\ \hline
Bulk               & Sine process     \\ \hline
\end{tabular}
\end{table}

\onslide<2>

\begin{table}[]
\begin{tabular}{|l|l|}
\hline
                   & GUE              \\ \hline
Biggest particule  & T.W              \\ \hline
Edge               & Soft edge (Airy) \\ \hline
Global convergence & Semi circular    \\ \hline
Fluctuations       & Gaussian         \\ \hline
Bulk               & Sine process     \\ \hline
\end{tabular}
\end{table}
Random matrices, OP ensembles DPP etc. 
    \begin{itemize}
        \item Global (semi circular law): 
        \cite{10.2307/1970008,Pastur1972}. 
        \item Local convergence (sine process): \cite*{MR2411912,Erdos2010}.
        \item Edge (Airy ensemble and Tracy-Widom fluctuations): \cite{tao2011}.
     
    \end{itemize}
\onslide<3>    
\begin{table}[]
\begin{tabular}{|l|l|}
\hline
                   & \begin{tabular}[c]{@{}l@{}}GUE + Wigner\\ (with a good control on moments)\end{tabular} \\ \hline
Extreme particle  & T.W                                                                                     \\ \hline
Edge               & Soft edge (Airy)                                                                        \\ \hline
Global convergence & Semi circular                                                                           \\ \hline
Fluctuations       & Gaussian                                                                                \\ \hline
Bulk               & Sine process                                                                            \\ \hline
\end{tabular}
\end{table}    
    
\onslide<4>    
    
    \footnotesize{
\begin{table}[]
\begin{tabular}{|l|l|l|}
\hline
                   & \begin{tabular}[c]{@{}l@{}}GUE + Wigner (with a good \\ control on moments)\end{tabular} & \begin{tabular}[c]{@{}l@{}}Uniform permutation \end{tabular} \\ \hline
Extreme particle  & T.W                                                                                        & T.W                                                                        \\ \hline
Edge               & Soft edge (Airy)                                                                           & Soft edge  (Airy)                                                          \\ \hline
Global convergence & Semi circular                                                                              & VKLS                                                                       \\ \hline
Fluctuations       & Gaussian  & Gaussian                                                                   \\ \hline
Bulk               & Sine process                                                                               & 
Discrete sine process                                                       \\ \hline
\end{tabular}
\end{table}}
\onslide<5>    
    
    \footnotesize{
\begin{table}[]
\begin{tabular}{|l|l|l|}
\hline
                   & \begin{tabular}[c]{@{}l@{}}GUE + Wigner (with a good \\ control on moments)\end{tabular} & \begin{tabular}[c]{@{}l@{}}Uniform permutation \end{tabular} \\ \hline
Extreme particle  & T.W                                                                                        & T.W                                                                        \\ \hline
Edge               & Soft edge (Airy)                                                                           & Soft edge  (Airy)                                                          \\ \hline
Global convergence & Semi circular                                                                              & VKLS                                                                       \\ \hline
Fluctuations       & Gaussian  & Gaussian                                                                   \\ \hline
Bulk               & Sine process                                                                               & Discrete sine process                                                       \\ \hline
\end{tabular}
\end{table}}
\large{Universality for  random permutations ?}
\large{Independence? Moments?}

\end{overprint}

\end{frame}


\section{Longest increasing subsequence and Ulam–Hammersley problem.}
\begin{frame}{Plan}
\tableofcontents[currentsection,currentsubsection,
    hideothersubsections, 
    sectionstyle=show/shaded,
]
\end{frame}

\begin{frame}{Longest increasing subsequence}
\begin{itemize}

\item $\mathfrak{S_n}$: symmetric group, (the  group of permutations of $\{1,\dots,n\}).$
\\ 
\item $(\sigma(i_1),\dots,\sigma(i_k))$  increasing  subsequence of $\sigma$ of length $k$ if $i_1<i_2<\dots<i_k$ and $\sigma(i_1)<\dots<\sigma(i_k)$.
\item $\ell(\sigma)$: the length of the longest increasing subsequence of $\sigma$.
\item For example: $$\sigma = \begin{pmatrix} 
1 & 2 & 3 & 4 & 5 & 6 & 7 & 8 \\
3 & 8 & \color{red} 1 &\color{red} 2 &\color{red} 4 & 7 &\color{red} 5 &\color{red} 6
  \end{pmatrix}.$$
$\ell(\sigma)=5$.
\end{itemize}
    \begin{conjecture}[\cite{ulam}]
    If $\sigma_n \sim {U}_{\mathfrak{S}_n}$, then
    $$\lim_{n\to \infty}\frac{\mathbb{E}(\ell(\sigma_n))}{\sqrt{n}}=c.$$ 
    \end{conjecture}
\end{frame}

\begin{frame}{Longest increasing subsequence}
\begin{theorem}[\cite{vershik,LOGAN}]
  If $\sigma_n \sim {U}_{\mathfrak{S}_n}$ then
    $$\lim_{n\to \infty}\frac{\mathbb{E}(\ell(\sigma_n))}{\sqrt{n}}=2$$
    and
    $$ \frac{\ell(\sigma_n)}{\sqrt{n}} \overset{\mathbb{P}}\to 2.$$ 
\end{theorem}
\begin{theorem} [\cite*{Baik1999}] \label{dbj}
 If $\sigma_n \sim {U}_{\mathfrak{S}_n}$ then
\begin{equation*} 
\lim_{n \to \infty} \mathbb{P}\left(\frac{\ell(\sigma_n)-2\sqrt{n}}{n^\frac 16}\leq s\right)=F_2(s).
\end{equation*}
\end{theorem}
    $F_2$: CDF of the GUE Tracy-Widom distribution. 
\end{frame}

\begin{frame}
\footnotesize{
\begin{table}[]
\begin{tabular}{|l|l|l|}
\hline
                   & \begin{tabular}[c]{@{}l@{}}GUE + Wigner (with a good \\ control on moments)\end{tabular} & \begin{tabular}[c]{@{}l@{}}Uniform permutation \end{tabular} \\ \hline
Extreme particle  & T.W                                                                                        & {\color{red}T.W}                                                                        \\ \hline
Edge               & Soft edge (Airy)                                                                           & Soft edge  (Airy)                                                          \\ \hline
Global convergence & Semi circular                                                                              & VKLS                                                                       \\ \hline
Fluctuations       & Gaussian  & Gaussian                                                                   \\ \hline
Bulk               & Sine process                                                                               & Discrete sine process                                                       \\ \hline
\end{tabular}
\end{table}}
\end{frame}
\begin{frame}{Longest increasing subsequence}
    \begin{theorem}[\cite{sk}]
Assume that the sequence of random permutations  $(\sigma_n)_{n\geq 1}$ satisfies:
\begin{itemize}
\item  For all positive integer $n$, $\sigma_n$ is invariant under conjugation i.e.  $\forall \sigma , \rho \in \mathfrak{S}_n$,
\begin{equation}\tag{H1}\label{h1}
\mathbb{P}(\sigma_n=\sigma)=\mathbb{P}(\sigma_n=\rho^{-1}\sigma\rho).
\end{equation}
\item The number of cycles is such that: For all $\varepsilon>0$,
\begin{equation}\tag{H2}\label{h2}
\lim_{n\to \infty}\mathbb{P}\left(\frac{\#(\sigma_n)}{n^\frac 16 }>\varepsilon\right) =0.
\end{equation}
\end{itemize}
Then  for all  $s \in \mathbb{R}$,
\begin{equation}\tag{TW}\label{TW} 
\lim_{n\to \infty} \mathbb{P}\left(\frac{\ell(\sigma_n)-2\sqrt{n}}{n^\frac 16}\leq s\right)=F_2(s).
\end{equation}
\end{theorem}
\end{frame}

\begin{frame}{Ewens' case}
\begin{definition}[Ewens distribution]
Let $\theta\geq 0$. If $\sigma_n\sim Ew(\theta)$ then
\begin{equation*}
\mathbb{P}(\sigma_n=\sigma)=\frac{\theta^{\#(\sigma)-1}}{\prod_{k=1}^{n-1}(\theta+k)}.\end{equation*}
\end{definition}
\vspace{5 mm}

\begin{itemize}
\item $\theta=1$:  uniform distribution.
\item $\theta=0$: uniform distribution on permutations with a unique cycle. 
\item $\mathbb{E}(\#(\sigma_n))= 1+\sum_{k=1}^{n-1} \frac{\theta}{\theta+k}\sim \theta \log(n).$
\end{itemize}
\end{frame}
\begin{frame}{Ewens' case}
\begin{corollary} \label{2.1}
Assume that $\sigma_n\sim Ew(\theta_n)$. If
\begin{equation}\tag{H'2}\label{ewcond}
\lim_{n\to \infty} \frac{\theta_n \log(n)  }{n^\frac 1 6}=0.
\end{equation}
Then 
\begin{equation}  \tag{TW}
\lim_{n\to \infty} \mathbb{P}\left(\frac{\ell(\sigma_n)-2\sqrt{n}}{n^\frac 16}\leq s\right)=F_2(s).
\end{equation}
\end{corollary}
\vspace{10 mm }
Other applications: Ewens-Pitman, virtual permutations (Kingman), etc.
\end{frame}

\begin{frame}{Proof}
We denote by:
\begin{itemize}
    \item $A_\sigma:=
    \begin{cases}
        \left\{\rho\in\mathfrak{S}_n, \sigma\circ\rho^{-1}=(i,j)  \text{ and }  \#(\rho)=\#(\sigma)-1 \right\} & \text{ if } \#(\sigma)>1
        \\ \{\sigma\} & \text{ if } \#(\sigma)=1
    \end{cases}.
    $
    \item $T$: Markov operator associated to  $\left[\frac{\mathbf{1}_{A_{\sigma_1}}(\sigma_2)}{card(A_{\sigma_1})}\right]_{\sigma_1,\sigma_2\in \mathfrak{S}_n}$.
\end{itemize}

\end{frame}
\begin{frame}{Proof}
\begin{figure}[H]
\centering
\begin{tikzpicture}

    \node[state] (s1)  {Id};
    \node[state, below=1cm of s1] (t1) {$(1,2)$};
    \node[state, right=2cm of t1] (t2) {$(2,3)$};
	\node[state, left=2cm of t1] (t3) {$(1,3)$};
	\node[state, below=1cm  of t3] (c1) {$(1,2,3)$};
	\node[state, right= 5cm of c1] (c2) {$(1,3,2)$};
        \draw[every loop,
        line width=0.3mm,
        auto=left,
        >=latex,
        ]
            (s1) edge[]  node {$\frac{1}{3}$} (t1)
             (s1) edge[]  node {$\frac{1}{3}$} (t3)
              (s1) edge[]  node {$\frac{1}{3}$} (t2)
                 (t1) edge[ ]  node {$\frac{1}{2}$} (c1)
       (t2) edge[ ]  node {$\frac{1}{2}$} (c1)
                 (t3) edge[ ]  node {$\frac{1}{2}$} (c1)
                 (t1) edge[]  node {$\frac{1}{2}$} (c2)
                 (t2) edge[]  node {$\frac{1}{2}$} (c2)
                 (t3) edge[]  node {$\frac{1}{2}$} (c2)
                 (c2) edge[loop below]  node {1} (c2)
                 (c1) edge[loop below]  node {1} (c1);
    \end{tikzpicture}
    \caption{The transition probabilities of $T$ on $\mathfrak{S}_3$}
    \label{figm}
\end{figure}
\end{frame}
\begin{frame}{Proof}
    \begin{itemize}
        \item If $\sigma_n$ is invariant under conjugation, $T(\sigma_n)$ is also invariant under conjugation.
        \item $
\#(T \left(\sigma_{n})\right)\overset{a.s}{=}\max(\#(\sigma_n)-1,1). $  
    \end{itemize}
    Consequently, if $\sigma_n$ is invariant under conjugation, then 
    \begin{itemize}
        \item If $\sigma_n$ is invariant under conjugation, $T^{n-1}(\sigma_n)$ is also invariant under conjugation.
        \item Almost surely, $\#\left(T^{n-1}(\sigma_n)\right)=1$.
    \end{itemize}
\end{frame}
\begin{frame}{Proof}
\begin{lemma} \label{lem}
 $\forall \sigma \in  \mathfrak{S}_n$,  $ \forall\tau=(i,j)$ a transposition, 
\begin{equation*}
|\ell(\sigma \circ \tau )-\ell(\sigma)|\leq 2.
\end{equation*}
\end{lemma}
\begin{lemma}
If
\begin{itemize}
    \item $\sigma_n$ is invariant under conjugation. 
    \item Almost surely, $\#(\sigma_n)=1$.
\end{itemize}
Then  $\sigma_n \sim Ew(0).$
\end{lemma}
\end{frame}

\begin{frame}{Proof}
If $\sigma_n$ is invariant under conjugation then
\begin{itemize}
    \item $T^{n-1}(\sigma_n)\sim Ew(0).$
    \item Almost surely, $$
|\ell(T^{n-1}(\sigma_n))-\ell(\sigma_n)|=
|\ell(T^{\#(\sigma_n)-1}(\sigma_n))-\ell(\sigma_n)|
\leq  2(\#(\sigma_n)-1).$$
\end{itemize}
Assume that  $\frac{\#(\sigma_n)}{n^\frac16} \overset{\mathbb{P}}\to 0 $.  We obtain $\frac{\ell(T^{n-1}(\sigma_n))-\ell(\sigma_n)}{n^\frac 16} \overset{\mathbb{P}}\to 0$.
\\ 

\eqref{TW}: Uniform $\Rightarrow$ $Ew(0)$ $\Rightarrow$  Invariant under conjugation +   $\frac{\#(\sigma_n)}{n^\frac16} \overset{\mathbb{P}}\to 0 $
\end{frame}
\section{The first arrows of random Young tableaux (edge)}

\begin{frame}{Plan}
\tableofcontents[currentsection,currentsubsection,
    hideothersubsections, 
    sectionstyle=show/shaded,
]
\end{frame}
\begin{frame}{Young diagram}
   \begin{definition}[Young diagram]
   $\lambda=(\lambda_i)_{i\geq1} \in \mathbb{N}^{\mathbb{N}^*}$ is a Young diagram of size $n$ if 
   \begin{itemize}
       \item $\forall i\geq1, \  \lambda_{i+1}\leq \lambda_i$,
       \item $\sum_{i=1}^\infty \lambda_i=n$.
   \end{itemize}
   \end{definition}
   \vspace{10 mm}

   Example:  Young diagrams of size 3 are   \\ $\mathbb{Y}_3=(3,\underline{0}),(2,1,\underline{0}),(1,1,1,\underline{0})$
   \\   or $\left(\yng(3),\yng(2,1),\yng(1,1,1)\right)$.
   
 
\end{frame}


\begin{frame}{Young tableau}
    \begin{definition}[Young tableau]
    A Young tableau of shape $\lambda$ is a filling of the boxes of $\lambda$ using the entries $\{1,2,\dots,n\}$ and the entries in each row and each column are increasing.
    \end{definition}
    \begin{itemize}
        \item     Example: Young tableaux of shape $\yng(3,1)$ are $\young(123,4),\young(124,3),\young(134,2)$. 
    \item$dim(\lambda)=$ \# Young tableaux of shape $\lambda$. 
    \item Example: $dim\left(\yng(3,1)\right)=3$.
    \item $dim(\lambda) = $   dimension of the irreducible representation of  $\mathfrak{S}_n$ indexed by $\lambda$.
     \item $\sum_{\lambda \in \mathbb{Y}_n} dim(\lambda)^2=\#(\mathfrak{S}_n)=n!$.
    \end{itemize}
 \end{frame}



\begin{frame}{Viennot's geometric construction
}
$$\sigma = \begin{pmatrix} 
1 & 2 & 3 & 4 & 5 & 6 & 7 & 8 \\
3 & 8 & 1 & 2 & 4 & 7 & 5 & 6
  \end{pmatrix}.$$

 \begin{overprint}
   \onslide<1> 

  \begin{center}
\begin{tikzpicture}    [/pgfplots/y=0.4cm, /pgfplots/x=0.4cm]
      \begin{axis}[
axis x line=center,
    axis y line=center,
    xmin=0, xmax=10,
    ymin=0, ymax=10, clip=false,
    xtick={0,1,2,3,3,4,5,6,7,8,9},
    ytick={0,1,2,3,3,4,5,6,7,8,9},
    grid=both,
    legend pos=north west,
    anchor=origin,
    grid style=dashed    ,
]
 \node[outer sep=0pt,circle, fill=red,inner sep=1.5pt] (P) at (1,3) {};
 \node[outer sep=0pt,circle, fill=red,inner sep=1.5pt] (P) at (2,8) {};
 \node[outer sep=0pt,circle, fill=red,inner sep=1.5pt] (P) at (3,1) {};
 \node[outer sep=0pt,circle, fill=red,inner sep=1.5pt] (P) at (4,2) {};
 \node[outer sep=0pt,circle, fill=red,inner sep=1.5pt] (P) at (5,4) {};
 \node[outer sep=0pt,circle, fill=red,inner sep=1.5pt] (P) at (6,7) {};
 \node[outer sep=0pt,circle, fill=red,inner sep=1.5pt] (P) at (7,5) {};
 \node[outer sep=0pt,circle, fill=red,inner sep=1.5pt] (P) at (8,6) {};

\end{axis}

    \end{tikzpicture}
\end{center}


   \onslide<2> 
   
  \begin{center}
\begin{tikzpicture}    [/pgfplots/y=0.4cm, /pgfplots/x=0.4cm]
      \begin{axis}[
axis x line=center,
    axis y line=center,
    xmin=0, xmax=10,
    ymin=0, ymax=10, clip=false,
    xtick={0,1,2,3,3,4,5,6,7,8,9},
    ytick={0,1,2,3,3,4,5,6,7,8,9},
    grid=both,
    legend pos=north west,
    anchor=origin,
    grid style=dashed    ,
]
 \node[outer sep=0pt,circle, fill=red,inner sep=1.5pt] (P) at (1,3) {};
 \node[outer sep=0pt,circle, fill=red,inner sep=1.5pt] (P) at (2,8) {};
 \node[outer sep=0pt,circle, fill=red,inner sep=1.5pt] (P) at (3,1) {};
 \node[outer sep=0pt,circle, fill=red,inner sep=1.5pt] (P) at (4,2) {};
 \node[outer sep=0pt,circle, fill=red,inner sep=1.5pt] (P) at (5,4) {};
 \node[outer sep=0pt,circle, fill=red,inner sep=1.5pt] (P) at (6,7) {};
 \node[outer sep=0pt,circle, fill=red,inner sep=1.5pt] (P) at (7,5) {};
 \node[outer sep=0pt,circle, fill=red,inner sep=1.5pt] (P) at (8,6) {};
\draw [fill=cyan,cyan] (1,3) rectangle (10,10);

\end{axis}

    \end{tikzpicture}
\end{center}
      \onslide<3>
     \begin{center}
\begin{tikzpicture}    [/pgfplots/y=0.4cm, /pgfplots/x=0.4cm]
      \begin{axis}[
axis x line=center,
    axis y line=center,
    xmin=0, xmax=10,
    ymin=0, ymax=10, clip=false,
    xtick={0,1,2,3,3,4,5,6,7,8,9},
    ytick={0,1,2,3,3,4,5,6,7,8,9},
    grid=both,
    legend pos=north west,
    anchor=origin,
    grid style=dashed    ,
]
 \node[outer sep=0pt,circle, fill=red,inner sep=1.5pt] (P) at (2,8) {};
 \node[outer sep=0pt,circle, fill=red,inner sep=1.5pt] (P) at (3,1) {};
 \node[outer sep=0pt,circle, fill=red,inner sep=1.5pt] (P) at (4,2) {};
 \node[outer sep=0pt,circle, fill=red,inner sep=1.5pt] (P) at (5,4) {};
 \node[outer sep=0pt,circle, fill=red,inner sep=1.5pt] (P) at (6,7) {};
 \node[outer sep=0pt,circle, fill=red,inner sep=1.5pt] (P) at (7,5) {};
 \node[outer sep=0pt,circle, fill=red,inner sep=1.5pt] (P) at (8,6) {};
\draw [fill=cyan,cyan] (1,3) rectangle (10,10);
\draw [fill=cyan,cyan] (3,1) rectangle (10,10);


\draw [line width=0.5mm,green] (1,3) -- (1,10);
\draw [line width=0.5mm,green] (1,3) -- (3,3);
\draw [line width=0.5mm,green] (3,3) -- (3,1);
\draw [line width=0.5mm,green] (3,1) -- (10,1);
 \node[outer sep=0pt,circle, fill=red,inner sep=1.5pt] (P) at (1,3) {};
 \node[outer sep=0pt,circle, fill=red,inner sep=1.5pt] (P) at (3,1) {};

 \node[outer sep=0pt,circle, fill=blue,inner sep=1.5pt] (P) at (3,3) {};
\end{axis}

    \end{tikzpicture}
\end{center}

      \onslide<4>
     \begin{center}
\begin{tikzpicture}    [/pgfplots/y=0.4cm, /pgfplots/x=0.4cm]
      \begin{axis}[
axis x line=center,
    axis y line=center,
    xmin=0, xmax=10,
    ymin=0, ymax=10, clip=false,
    xtick={0,1,2,3,3,4,5,6,7,8,9},
    ytick={0,1,2,3,3,4,5,6,7,8,9},
    grid=both,
    legend pos=north west,
    anchor=origin,
    grid style=dashed    ,
]
 \node[outer sep=0pt,circle, fill=red,inner sep=1.5pt] (P) at (1,3) {};
 \node[outer sep=0pt,circle, fill=red,inner sep=1.5pt] (P) at (2,8) {};
 \node[outer sep=0pt,circle, fill=red,inner sep=1.5pt] (P) at (3,1) {};
 \node[outer sep=0pt,circle, fill=red,inner sep=1.5pt] (P) at (4,2) {};
 \node[outer sep=0pt,circle, fill=red,inner sep=1.5pt] (P) at (5,4) {};
 \node[outer sep=0pt,circle, fill=red,inner sep=1.5pt] (P) at (6,7) {};
 \node[outer sep=0pt,circle, fill=red,inner sep=1.5pt] (P) at (7,5) {};
 \node[outer sep=0pt,circle, fill=red,inner sep=1.5pt] (P) at (8,6) {};

\draw [line width=0.5mm,green] (1,3) -- (1,10);
\draw [line width=0.5mm,green] (1,3) -- (3,3);
\draw [line width=0.5mm,green] (3,3) -- (3,1);
\draw [line width=0.5mm,green] (3,1) -- (10,1);
 \node[outer sep=0pt,circle, fill=red,inner sep=1.5pt] (P) at (1,3) {};
 \node[outer sep=0pt,circle, fill=red,inner sep=1.5pt] (P) at (3,1) {};

 \node[outer sep=0pt,circle, fill=blue,inner sep=1.5pt] (P) at (3,3) {};
\end{axis}

    \end{tikzpicture}
\end{center}
     \onslide<5>
     \begin{center}
\begin{tikzpicture}    [/pgfplots/y=0.4cm, /pgfplots/x=0.4cm]
      \begin{axis}[
axis x line=center,
    axis y line=center,
    xmin=0, xmax=10,
    ymin=0, ymax=10, clip=false,
    xtick={0,1,2,3,3,4,5,6,7,8,9},
    ytick={0,1,2,3,3,4,5,6,7,8,9},
    grid=both,
    legend pos=north west,
    anchor=origin,
    grid style=dashed    ,
]
 \node[outer sep=0pt,circle, fill=red,inner sep=1.5pt] (P) at (1,3) {};
 \node[outer sep=0pt,circle, fill=red,inner sep=1.5pt] (P) at (2,8) {};
 \node[outer sep=0pt,circle, fill=red,inner sep=1.5pt] (P) at (3,1) {};
 \node[outer sep=0pt,circle, fill=red,inner sep=1.5pt] (P) at (4,2) {};
 \node[outer sep=0pt,circle, fill=red,inner sep=1.5pt] (P) at (5,4) {};
 \node[outer sep=0pt,circle, fill=red,inner sep=1.5pt] (P) at (6,7) {};
 \node[outer sep=0pt,circle, fill=red,inner sep=1.5pt] (P) at (7,5) {};
 \node[outer sep=0pt,circle, fill=red,inner sep=1.5pt] (P) at (8,6) {};

\draw [line width=0.5mm,green] (1,3) -- (1,10);
\draw [line width=0.5mm,green] (1,3) -- (3,3);
\draw [line width=0.5mm,green] (3,3) -- (3,1);
\draw [line width=0.5mm,green] (3,1) -- (10,1);
 \node[outer sep=0pt,circle, fill=red,inner sep=1.5pt] (P) at (1,3) {};
 \node[outer sep=0pt,circle, fill=red,inner sep=1.5pt] (P) at (3,1) {};

 \node[outer sep=0pt,circle, fill=blue,inner sep=1.5pt] (P) at (3,3) {};
 
 
 \draw [fill=cyan,cyan] (2,8) rectangle (10,10);
\draw [fill=cyan,cyan] (4,2) rectangle (10,10);


\draw [line width=0.5mm,green] (2,8) -- (2,10);
\draw [line width=0.5mm,green] (2,8) -- (4,8);
\draw [line width=0.5mm,green] (4,8) -- (4,2);
\draw [line width=0.5mm,green] (4,2) -- (10,2);
 \node[outer sep=0pt,circle, fill=red,inner sep=1.5pt] (P) at (2,8) {};
 \node[outer sep=0pt,circle, fill=red,inner sep=1.5pt] (P) at (4,2) {};

 \node[outer sep=0pt,circle, fill=blue,inner sep=1.5pt] (P) at (4,8) {};
\end{axis}

    \end{tikzpicture}
\end{center}




     \onslide<6>
     \begin{center}
\begin{tikzpicture}    [/pgfplots/y=0.4cm, /pgfplots/x=0.4cm]
      \begin{axis}[
axis x line=center,
    axis y line=center,
    xmin=0, xmax=10,
    ymin=0, ymax=10, clip=false,
    xtick={0,1,2,3,3,4,5,6,7,8,9},
    ytick={0,1,2,3,3,4,5,6,7,8,9},
    grid=both,
    legend pos=north west,
    anchor=origin,
    grid style=dashed    ,
]


\draw [line width=0.5mm,green] (2,8) -- (2,10);
\draw [line width=0.5mm,green] (2,8) -- (4,8);
\draw [line width=0.5mm,green] (4,8) -- (4,2);
\draw [line width=0.5mm,green] (4,2) -- (10,2);
 \draw [line width=0.5mm,green] (5,4) -- (5,10);
\draw [line width=0.5mm,green] (5,4) -- (10,4);

 \draw [line width=0.5mm,green] (8,6) -- (8,10);
\draw [line width=0.5mm,green] (8,6) -- (10,6);
\draw [line width=0.5mm,green] (1,3) -- (1,10);
\draw [line width=0.5mm,green] (1,3) -- (3,3);
\draw [line width=0.5mm,green] (3,3) -- (3,1);
\draw [line width=0.5mm,green] (3,1) -- (10,1);

\draw [line width=0.5mm,green] (6,7) -- (6,10);
\draw [line width=0.5mm,green] (6,7) -- (7,7);
\draw [line width=0.5mm,green] (7,5) -- (7,7);
\draw [line width=0.5mm,green] (7,5) -- (10,5);

 \node[outer sep=0pt,circle, fill=red,inner sep=1.5pt] (P) at (1,3) {};
 \node[outer sep=0pt,circle, fill=red,inner sep=1.5pt] (P) at (2,8) {};
 \node[outer sep=0pt,circle, fill=red,inner sep=1.5pt] (P) at (3,1) {};
 \node[outer sep=0pt,circle, fill=red,inner sep=1.5pt] (P) at (4,2) {};
 \node[outer sep=0pt,circle, fill=red,inner sep=1.5pt] (P) at (5,4) {};
 \node[outer sep=0pt,circle, fill=red,inner sep=1.5pt] (P) at (6,7) {};
 \node[outer sep=0pt,circle, fill=red,inner sep=1.5pt] (P) at (7,5) {};
 \node[outer sep=0pt,circle, fill=red,inner sep=1.5pt] (P) at (8,6) {};

 \node[outer sep=0pt,circle, fill=blue,inner sep=1.5pt] (P) at (3,3) {};
 \node[outer sep=0pt,circle, fill=blue,inner sep=1.5pt] (P) at (4,8) {};
  \node[outer sep=0pt,circle, fill=blue,inner sep=1.5pt] (P) at (7,7) {};


\end{axis}

    \end{tikzpicture}
\end{center}
$$\young(12456),\young(12568)$$



     \onslide<7>
     \begin{center}
\begin{tikzpicture}    [/pgfplots/y=0.4cm, /pgfplots/x=0.4cm]
      \begin{axis}[
axis x line=center,
    axis y line=center,
    xmin=0, xmax=10,
    ymin=0, ymax=10, clip=false,
    xtick={0,1,2,3,3,4,5,6,7,8,9},
    ytick={0,1,2,3,3,4,5,6,7,8,9},
    grid=both,
    legend pos=north west,
    anchor=origin,
    grid style=dashed    ,
]

\draw [line width=0.5mm,green] (3,3) -- (3,10);
\draw [line width=0.5mm,green] (3,3) -- (10,3);
\draw [line width=0.5mm,green] (4,8) -- (4,10);
\draw [line width=0.5mm,green] (4,8) -- (7,8);
\draw [line width=0.5mm,green] (7,7) -- (7,8);
\draw [line width=0.5mm,green] (7,7) -- (10,7);

 \node[outer sep=0pt,circle, fill=blue,inner sep=1.5pt] (P) at (3,3) {};
 \node[outer sep=0pt,circle, fill=blue,inner sep=1.5pt] (P) at (4,8) {};
  \node[outer sep=0pt,circle, fill=blue,inner sep=1.5pt] (P) at (7,7) {};
  \node[outer sep=0pt,circle, fill=black,inner sep=1.5pt] (P) at (7,8) {};


\end{axis}

    \end{tikzpicture}
\end{center}
$$\young(12456,37),\young(12568,34)$$

     \onslide<8>
     \begin{center}
\begin{tikzpicture}    [/pgfplots/y=0.4cm, /pgfplots/x=0.4cm]
      \begin{axis}[
axis x line=center,
    axis y line=center,
    xmin=0, xmax=10,
    ymin=0, ymax=10, clip=false,
    xtick={0,1,2,3,3,4,5,6,7,8,9},
    ytick={0,1,2,3,3,4,5,6,7,8,9},
    grid=both,
    legend pos=north west,
    anchor=origin,
    grid style=dashed    ,
]

\draw [line width=0.5mm,green] (10,8) -- (7,8);
\draw [line width=0.5mm,green] (7,10) -- (7,8);

  \node[outer sep=0pt,circle, fill=black,inner sep=1.5pt] (P) at (7,8) {};


\end{axis}

    \end{tikzpicture}
\end{center}
$$\young(12456,37,8),\young(12568,34,7) $$

    \end{overprint} 

\end{frame}

\begin{frame}{Robinson-Schensted correspondence}
\begin{itemize}
 
    \item One-to-one correspondence between permutations and pairs of standard Young tableaux of the same shape. 
    \item We denote by $\lambda(\sigma):=(\lambda_i(\sigma))_{i\geq1}$ the shape of the image of  $\sigma$ by this correspondence. 
    For example, if $$\sigma = \begin{pmatrix} 
1 & 2 & 3 & 4 & 5 & 6 & 7 & 8 \\
3 & 8 & 1 & 2 & 4 & 7 & 5 & 6
  \end{pmatrix}  \quad \text { then } \quad \lambda(\sigma)=\yng(5,2,1).$$ 

    \item $\ell(\sigma)=\lambda_1(\sigma).$
\end{itemize}    
\end{frame}

\begin{frame}{Plancherel measure}
\vspace{4 mm}

\begin{itemize}
    \item If $\sigma_n \sim {U}_{\mathfrak{S}_n}$ then $ \lambda(\sigma_n) \sim PL_n$. For any $\mu \in \mathbb{Y}_n$,
    \begin{align*}
    \mathbb{P}(\lambda(\sigma_n)=\mu)&=\frac{\#\{\text{pairs of Young tableaux of shape } \mu\}}{C}\\&=\frac{dim(\mu)^2}{n!}. 
    \end{align*}
    \item The poissonized version. If $ \lambda \sim PL^\theta$ then  for any $\mu \in \cup_{n\geq 1}\mathbb{Y}_n$,
$$    \mathbb{P}(\lambda_\theta=\mu) = e^{-\theta}\frac{\theta^{|\mu|}dim(\mu)^2}{|\mu|!^2}.$$
\item  If  $\lambda \sim PL^\theta$ then $\{\lambda_i-i\}_{i\geq1}$ is determinantal with discreet Bessel kernel. 
\end{itemize}

\end{frame}

\begin{frame}{Edge: Plancherel case}
\begin{theorem} [\cite*{Borodin2000}]
 If $\sigma_n \sim {U}_{\mathfrak{S}_n}$ then  $\forall k \geq 1 $,  $\forall s_1,s_2,\dots,s_k \in \mathbb{R}$,
\begin{equation*} 
\lim_{n\to \infty}\mathbb{P}\left(\forall i\leq k, \;\frac{\lambda_i(\sigma_n)-2\sqrt{n}}{n^\frac{1}{6}}\leq s_i\right)=\mathbb{P}(\forall i\leq k,\;\xi_i\leq s_i).
\end{equation*}
$\{\xi_1\geq \xi_2\geq\dots\geq  \xi_k \geq  \dots \}$: Airy ensemble.
\end{theorem}

\end{frame}
\begin{frame}{Edge: generalization}
    \begin{theorem}[\cite{sk}]
Assume that the sequence of random permutations  $(\sigma_n)_{n\geq 1}$ satisfies:
\begin{itemize}
\item  For all positive integer $n$, $\sigma_n$ is invariant under conjugation i.e.  $\forall \sigma , \rho \in \mathfrak{S}_n$,
\begin{equation}\tag{H1}
\mathbb{P}(\sigma_n=\sigma)=\mathbb{P}(\sigma_n=\rho^{-1}\sigma\rho).
\end{equation}
\item The number of cycles is such that: For all $\varepsilon>0$,
\begin{equation}\tag{H2}
\lim_{n\to \infty}\mathbb{P}\left(\frac{\#(\sigma_n)}{n^\frac 16 }>\varepsilon\right) =0.
\end{equation}
\end{itemize}
Then  for all  $s \in \mathbb{R}$,
\begin{equation*} 
\lim_{n\to \infty}\mathbb{P}\left(\forall i\leq k, \;\frac{\lambda_i(\sigma_n)-2\sqrt{n}}{n^\frac{1}{6}}\leq s_i\right)=\mathbb{P}(\forall i\leq k,\;\xi_i\leq s_i).\end{equation*}
\end{theorem}
\end{frame}

\begin{frame}{Greene's theorem}
    We denote by  \begin{align*}
\mathfrak{I}_1(\sigma):&=\{s\subset\{1,2,\dots,n\};\; \forall i,j \in s,\; (i-j)(\sigma(i)-\sigma(j))\geq 0 \},
\\\mathfrak{I}_{k+1}(\sigma):&=\{s\cup s',\; s\in \mathfrak{I}_k,\;s'\in \mathfrak{I}_1\}.
\end{align*}
We have then
\begin{lemma}[\cite{GREENE1974254}]
For any permutation $ \sigma\in \mathfrak{S}_n$,
\begin{align*}
\max_{s\in \mathfrak{I}_i(\sigma)} |s| =\sum_{k=1}^i \lambda_k(\sigma).
\end{align*}
\end{lemma}
In particular, $$\max_{s\in \mathfrak{I}_1(\sigma)} |s| =\lambda_1(\sigma)=\ell(\sigma).$$
\end{frame}
\section{The Vershik-Kerov-Logan-Shepp shape (global convergence)}
\begin{frame}{Proof}
\begin{lemma} \label{lemma2}
For any permutation $\sigma$ and transposition  $\tau$,\begin{equation*} \label{sum}
\left|\sum_{k=1}^i \lambda_k(\sigma)-{\lambda}_k\left(\sigma\circ\tau\right)\right| \leq 2
\end{equation*}
Consequently, 
\begin{equation*} \label{sep}
\left|\lambda_i(\sigma)-\lambda_i\left(\sigma\circ\tau\right)\right| \leq 4.
\end{equation*}
\end{lemma}
\begin{corollary}
\begin{equation} \label{sep2}
\left|\lambda_i(\sigma_n)-\lambda_i\left(T^{n-1}(\sigma_n)\right)\right| \leq 4(\#(\sigma_n)-1).
\end{equation}
\end{corollary}
\end{frame}
\begin{frame}{Plan}
\tableofcontents[currentsection,currentsubsection,
    hideothersubsections, 
    sectionstyle=show/shaded,
]
\end{frame}



\begin{frame}{Russian notations}
\begin{itemize}
    \item Rotate the diagram by $\frac{3\pi}{4}$.
    \item Complete the high function by  $x\to|x|$.
    \item We denote by $L_\lambda$ the resulting function.
\end{itemize}
\begin{figure}
\centering
\begin{tikzpicture}    [/pgfplots/y=0.3cm, /pgfplots/x=0.3cm]
      \begin{axis}[
    axis x line=center,
    axis y line=center,
    xmin=0, xmax=10,
    ymin=0, ymax=10, clip=false,
    ytick={0},
	xtick={0},
    minor xtick={0,1,2,3,3,4,5,6,7,8,9},
    minor ytick={0,1,2,3,3,4,5,6,7,8,9},
    grid=both,
    legend pos=north west,
    ymajorgrids=false,
    xmajorgrids=false, anchor=origin,
    grid style=dashed    , rotate around={45:(rel axis cs:0,0)}
,
]

\addplot[
    color=blue,
        line width=3pt,
    ]
    coordinates {
    (0,10)(0,5)(1,5)(1,2)(2,2)(2,1)(3,1)(3,0)(10,0)
    };
 
\end{axis}
\begin{axis}[
	axis x line=center,
    axis y line=center,
    xmin=-7.07, xmax=7.07,
    ymin=0, ymax=8, anchor=origin, clip=false,
    xtick={-7,-5,-3,-1,1,3,5,7},
    ytick={0,2,4,6,8},
    legend pos=north west,
    ymajorgrids=false,
    xmajorgrids=false,rotate around={0:(rel axis cs:0,0)},
    grid style=dashed];
\end{axis}
    \end{tikzpicture}
    \caption{ $L_{(5,2,1,\underline{0})}$}
     \label{figL}
\end{figure} 
\end{frame}



\begin{frame}{Vershik-Kerov-Logan-Shepp shape}
    \begin{theorem}[\cite{vershik,LOGAN}]
    If $\sigma_n \sim U_{\mathfrak{S}_n}$, then for any $\varepsilon>0$,
\begin{align*}
\lim_{n\to \infty} \mathbb{P}\left(\sup_{s\in \mathbb{R}} \left|\frac{1}{\sqrt{2n}}L_{\lambda(\sigma_n)}\left({s}{\sqrt{2n}}\right)-\Omega(s)\right|<\varepsilon\right) =1,
\end{align*}
where
\begin{align*}
\Omega(s):=\begin{cases}
\frac{2}{\pi}(s\arcsin({s})+\sqrt{1-s^2}) & \text{ if } |s|<1 \\ 
|s| & \text{ if } |s|\geq 1 
\end{cases}.
\end{align*}
    \end{theorem}
    \vspace{10 mm}

     $\Omega$ is strongly related to the semi-circular law.
\end{frame}
\begin{frame}{Vershik-Kerov-Logan-Shepp shape}
    \begin{figure}[ht]
    \centering
     \def\svgscale{.65}
    \input{diag.pdf_tex}
       

    \caption{Typical Young diagram under the Plancherel distribution}
\end{figure}

\end{frame}


 \begin{frame}{Limit shape}
    \begin{theorem}[\cite{sk}]
Assume that the sequence of random permutations  $(\sigma_n)_{n\geq 1}$ satisfies:
\begin{itemize}
\item  For all positive integer $n$, $\sigma_n$ is invariant under conjugation i.e.  $\forall \sigma , \rho \in \mathfrak{S}_n$,
\begin{equation}\tag{H1}
\mathbb{P}(\sigma_n=\sigma)=\mathbb{P}(\sigma_n=\rho^{-1}\sigma\rho).
\end{equation}
\item The number of cycles is such that: For all $\varepsilon>0$,
\begin{equation}\tag{H3}
\lim_{n\to \infty}\mathbb{P}\left(\frac{\#(\sigma_n)}{n}>\varepsilon\right) =0.
\end{equation}
\end{itemize}
Then  for all  $\varepsilon>0$,
\begin{align*}
\lim_{n\to \infty} \mathbb{P}\left(\sup_{s\in \mathbb{R}} \left|\frac{1}{\sqrt{2n}}L_{\lambda(\sigma_n)}\left({s}{\sqrt{2n}}\right)-\Omega(s)\right|<\varepsilon\right) =1.
\end{align*}
\end{theorem}
\end{frame}

\begin{frame}{Proof}

\begin{lemma} \label{34}
Let $n,m\in \mathbb{N}^*$, $\lambda=(\lambda_i)_{i\geq1}\in\mathbb{Y}_n$, $\mu=(\mu_i)_{i\geq1}\in\mathbb{Y}_m$. Then,
\begin{align*}\label{disVC}
||L_\lambda-L_\mu||_{\infty}^2 \leq 4 \max_{i\geq 1} \left|\sum_{k=1}^i( \lambda_k-\mu_k)\right|.
\end{align*}
Consequently, 
\begin{equation*}
\sup_{s\in \mathbb{R}} \frac{1}{\sqrt{2n}}\left|L_{\lambda(\sigma_n)}\left({s}{\sqrt{2n}}\right)- L_{\lambda(T^{n-1}(\sigma_n))}\left({s}{\sqrt{2n}}\right)\right| \leq 2 \sqrt{\frac{\#(\sigma_n)-1}{n}}.
\end{equation*}
\end{lemma}
\end{frame}
\begin{frame}{Application: Longest commun subsequence}
\begin{itemize}
\item $(\sigma(i_1),\dots,\sigma(i_k))$ subsequence of $\sigma$ of length $k$ if $ i_1<i_2<,\dots,<i_k$.
\item $LCS(\sigma_1,\sigma_2)$ the length of the longest common subsequence of two permutations.
\end{itemize}

\begin{conjecture}[\cite{MR3509473}]
 For any integer $n\geq 1$,  for any $\sigma_{1,n}$ and $\sigma_{2,n}$   independent and identically distributed  random permutations,
\begin{align*}
    \mathbb{E}(LCS(\sigma_{1,n},\sigma_{2,n}))\geq \sqrt{n}.
\end{align*}
\end{conjecture}
\end{frame}
\begin{frame}{Application: Longest common subsequence}


\begin{theorem}\label{thmp}
For any $0\leq \alpha<2$, there exists $n_\alpha\geq 1$ such that for any $n>n_\alpha$, for any $\sigma_{1,n}$ and $\sigma_{2,n}$   independent and identically distributed  random permutations with distribution invariant under conjugation.
\begin{align*}
    \mathbb{E}(LCS(\sigma_{1,n},\sigma_{2,n}))\geq \alpha\sqrt{n}.
\end{align*}
\end{theorem}

  
\end{frame}
\begin{frame}{Conclusion}
% Please add the following required packages to your document preamble:
% \usepackage[table,xcdraw]{xcolor}
% If you use beamer only pass "xcolor=table" option, i.e. \documentclass[xcolor=table]{beamer}
% Please add the following required packages to your document preamble:
% \usepackage[table,xcdraw]{xcolor}
% If you use beamer only pass "xcolor=table" option, i.e. \documentclass[xcolor=table]{beamer}
\scriptsize{
\begin{table}[]
\begin{tabular}{|l|l|l|l|}
\hline
                   & \begin{tabular}[c]{@{}l@{}}GUE\\ + Wigner (with a good \\ control on moments)\end{tabular} & Plancherel                                                      & \begin{tabular}[c]{@{}l@{}}Random permutations \\ (with a good control\\  on cycles' number)\end{tabular} \\ \hline
Extreme particle  & T.W                                                                                          & T.W                                                             & \cellcolor[HTML]{34FF34}T.W                                                                                 \\ \hline
Edge               & Soft edge (Airy)                                                                             & \begin{tabular}[c]{@{}l@{}}Soft edge \\ (Airy)\end{tabular}     & \cellcolor[HTML]{34FF34}Soft edge (Airy)                                                                    \\ \hline
Global convergence & Semi circular                                                                                & VKLS                                                            & \cellcolor[HTML]{34FF34}VKLS                                                                                \\ \hline
Fluctuations       & Gaussian                                                                                  & Gaussian                                                        & {\color[HTML]{FE0000} ??}                                                                                   \\ \hline
Bulk               & Sine process                                                                                 & \begin{tabular}[c]{@{}l@{}}Discrete sine \\ process\end{tabular} & {\color[HTML]{FE0000} ??}                                                                                   \\ \hline
\end{tabular}
\end{table}
}
\end{frame}

\begin{frame}{Conjectures}
\begin{itemize}
\item We need only $\frac{\#(\sigma_n)}{n^\frac{1}{2}} \overset{\mathbb{P}}\to 0 $ to obtain Tracy-Widom fluctuations. 
\item For any sequence of random permutations invariant under conjugation, for any $\varepsilon>0$
\begin{align*}
\lim_{n\to \infty} \mathbb{P}\left(\sup_{s\in \mathbb{R}} \left|\frac{1}{2\sqrt{n}}L_{\lambda(\sigma_n)}\left({2s\sqrt{n-fix(\sigma_n)}}\right)-\sqrt{1-\frac{fix(\sigma_n)}{n}}\Omega\left(s\right)\right|<\varepsilon\right) =1.
\end{align*}
\item Under a good control on cycles we have discrete sine process (Bulk).
\end{itemize}

\end{frame}

\begin{frame}

  \begin{columns}
    \begin{column}{0.70\textwidth}
      \begin{center}
      \begin{figure}[ht]
    \centering
        \def\svgwidth{\columnwidth}
    \input{diag.pdf_tex}
\end{figure}

      \end{center}
    \end{column}
    \begin{column}{0.30\textwidth}
      \begin{center}

        \font\endfont = cmss10 at 5.40mm
\color[rgb]{0.00,0.00,1.00}       \endfont 
        \baselineskip 7.0mm

        Merci de votre attention

      \end{center}    

    \end{column}
  \end{columns}

\end{frame}
\begin{frame}{Références}
\tiny

\bibliographystyle{abbrvnat}
    \bibliography{test}

\end{frame}

\end{document}